\chapter{Функциональная схема}

	За основу реализации была взята основная схема алгоритма работы внешнего дисплея.

	\includeimage
	{05-circuit}{f}{H}{0.8\textwidth}
	{Функциональная схема монитора.} % Подпись рисунка
	
	В устройстве используется декодер ''FT232L'', отладочная плата ''STM32F4DISCOVERY'' и дисплеи ''ST7735'', ''ILI9488''. Декодер подключается к компьютеру.
	
	Соединение монитора с компьютером осуществляется с помощью интерфейса передачи данных USB. Для модуля RS232 используется библиотека ''libftd2xx''. Декодер FT232L. Данные передаются в микроконтроллер по интерфейсу USART. В проекте используется микроконтроллер ''stm32f407vg''. В устройстве возможно использовать различные дисплеи от ST или ILI. Я использовал два дисплея ST7735 и ILI9488. Первый работал в режиме RGB565, то есть 16-bit. Второй в режиме RGB666, 18-bit.
	
%	Схема на рисунке \ref{img:05-My_algorithm_for_displaying_frames_on_the_display} описывает обший алгоритм передачи кадров на дисплей. Соединение монитора с компьютером осуществляется с помощью интерфейса передачи данных USB. Для модуля RS232 используется библиотека ''libftd2xx''. Декодер FT232L. Данные передаются в микроконтроллер по интерфейсу USART. В проекте используется микроконтроллер ''stm32f407vg''. В устройстве возможно использовать различные дисплеи от ST или ILI. Я использовал два дисплея ST7735 и ILI9488. Первый работал в режиме RGB565, то есть 16-bit. Второй в режиме RGB666, 18-bit.

