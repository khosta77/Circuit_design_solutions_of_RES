\chapter{Функциональная схема}

	\includeimage{05-circuit-work-device}{f}{H}{1.0\textwidth}
	{Функциональная схема устройства для теста дисплев.} % Подпись рисунка

	Идея работы устройства заключается в том, что с компьютера поступают данные о выводимом кадре, через декодер FT232L они передаются в микроконтроллер по универсальному синхронному приёмопередатчик. В Микроконтроллере DMA модуль записывает в одну из ячеек памяти считанный сектор. После для дисплея происходит настройка на обновление кадра. На порт ''DC'' поступает сигнал равные логической еденице, по последовательному периферийному интерфейсу поступает команда об обновлении кадра. И следом приходит команда о выводе координаты по оси ОХ. На порт DC поступает сигнал равные логическому нулю, По последовательному периферийному интерфейсу поступает левая и правая координата для оси ОХ. Аналогичным образом происходит настройка координат оси ОY. После происходит подтверждение начала передачи путем отправки соответствующей команды, при отправлении команды на выводе ''DC'' логическая единица. После на порт ''DC'' подается логический ноль и начинается передача сектора из памяти. 
	
	Пока происходит передача сектора, то в это время в другую ячейку памяти происходит запись следующего сектора. Эти операции происходят в цикле пока не будет вывед весь кадр.
	
	Так как дисплеи ILI и ST имеют одинаковый интерфейс взаимодействия, то не важно какой дисплей будет использоватся.

%	За основу реализации была взята основная схема алгоритма работы внешнего дисплея.

%	\includeimage{05-circuit}{f}{H}{0.8\textwidth}
%	{Функциональная схема монитора.} % Подпись рисунка
	
%	В устройстве используется декодер ''FT232L'', отладочная плата ''STM32F4DISCOVERY'' и дисплеи ''ST7735'', ''ILI9488''. Декодер подключается к компьютеру.
	
%	Соединение монитора с компьютером осуществляется с помощью интерфейса передачи данных USB. Для модуля RS232 используется библиотека ''libftd2xx''. Декодер FT232L. Данные передаются в микроконтроллер по интерфейсу USART. В проекте используется микроконтроллер ''stm32f407vg''. В устройстве возможно использовать различные дисплеи от ST или ILI. Я использовал два дисплея ST7735 и ILI9488. Первый работал в режиме RGB565, то есть 16-bit. Второй в режиме RGB666, 18-bit.
	
%	Схема на рисунке \ref{img:05-My_algorithm_for_displaying_frames_on_the_display} описывает обший алгоритм передачи кадров на дисплей. Соединение монитора с компьютером осуществляется с помощью интерфейса передачи данных USB. Для модуля RS232 используется библиотека ''libftd2xx''. Декодер FT232L. Данные передаются в микроконтроллер по интерфейсу USART. В проекте используется микроконтроллер ''stm32f407vg''. В устройстве возможно использовать различные дисплеи от ST или ILI. Я использовал два дисплея ST7735 и ILI9488. Первый работал в режиме RGB565, то есть 16-bit. Второй в режиме RGB666, 18-bit.

