\chapter{Функциональная и принципиальная схема}

	За основу реализации была взята основная схема алгоритма работы внешнего дисплея. 

	\includeimage
	{05-My_algorithm_for_displaying_frames_on_the_display}{f}{H}{0.6\textwidth}
	{Алгоритм обновления кадров в моем устройстве} % Подпись рисунка
	
	Схема на рисунке \ref{img:05-My_algorithm_for_displaying_frames_on_the_display} описывает обший алгоритм передачи кадров на дисплей. Соединение монитора с компьютером осуществляется с помощью интерфейса передачи данных USB. Для модуля RS232 используется библиотека ''libftd2xx''. Декодер FT232L. Данные передаются в микроконтроллер по интерфейсу USART. В проекте используется микроконтроллер ''stm32f407vg''. В устройстве возможно использовать различные дисплеи от ST или ILI. Я использовал два дисплея ST7735 и ILI9488. Первый работал в режиме RGB565, то есть 16-bit. Второй в режиме RGB666, 18-bit.

