\chapter{Алгоритм работы микроконтроллера}

	В предыдудущей главе была описана общая схема работы вывода кадров на секторы дисплея. В данной главе в начале будет описана схема инициализации микроконтроллера для работы, после "<тонкости"> работы вывода кадров.
	
	\textbf{Инициализация микроконтроллера}
	
	Производится настройка частоты работы. Ядро настраивантся на частоту 168 МГц, шина AHB1/APB1 на частоту 82 МГц, AHB2/APB2 42 МГц. После с помощью системной функции "<SystemCoreClock\_Update()"> происходит обновление переменной "<SystemCoreClock">, которая хранит в себе значение текущей частоты работы ядра.
	
	Производится настройка перифирии. Инициализируются порты входа и выхода. Порты, которые будут работать по стандартам универсального синхронного приёмопередатчика, в микроконтроллере USART2 и последовательному периферийному интерфейсу, в микроконтроллере SPI1 настраиваются на альтернативные функции, соответствующих режимов работы. Модуль универсального синхронного приёмопередатчика настраивается на чтения и запись, возможность считывать и передавать данные через DMA канал. Задается его частота работы по формуле:
	
	\[\text{Baud} = \frac{\text{SystemCoreClock}}{2 \cdot 460900}\]
	
	Происходит включение данного модуля. Следом настраивается DMA канал на прослушивание в не циклическом режиме, со сбратыванием прерывания по приему данных. 
	
	Модуль последовательного периферийного интерфейса настраивается на режим записи по DMA, в режиме "<мастера">, деление частоты скорости остаются по умолчанию равной одной второй. Произведя настройку, описанную выше, программа входит в бесконечный цикл, в состояние ожидания команды на обновление экрана дисплея.
	
	\textbf{Особенность работы алгоритма обновления кадра.}
	
	Алгоритм, описанный в предыдущей главе работает аналогичным образом в данном проекте. Но так как поддерживается большой набор дисплеев, то у них всех есть отличия в разрешении  формата цвета. То есть некоторые дисплеи работают с цветом в формате RGB565, а некоторые только с цветом в формате RGB666. Из-за этого возникают разногласия, если делать универсальный вывод. Для того, чтобы решить такую проблему было решено указывать при компиляции проекта какой дисплей будет подключаться, чтобы исключить возможность вывода искаженных кадров на дисплей.
	
	Готовый код проекта выложен в открытом доступе на сайте github в отдельном репозитории для физического устройства(\href{https://github.com/khosta77/stm32f4_ILI9488}{https://github.com/khosta77/stm32f4\_ILI9488}). Для драйвера( \href{https://github.com/khosta77/ft2xx_fast_start}{https://github.com/khosta77/ft2xx\_fast\_start}) на компьютер.


%	В предыдудущей главе была описана общая схема работы монитора. Но в начале разберем общие настройки и инициалиазию микроконтроллера и модуля RS232.
	
%	\textbf{Инициализация микроконтроллера.}
	
%	\begin{enumerate}
%		\item Настройка частоты работы. Ядро настраивается на частоту 168 МГц, шина AHB/APB1 на частоту 82 МГц, AHB/APB2 42 МГц.
%		\item Инициализация USART2:
%		\begin{enumerate}
%			\item Получение системной частоты.
%			\item Настройка интерфейса GPIO.
%			\item Настройка USART на чтение и запись.
%			\item Настройка DMA канала.
%			\item Включения прерываний на чтение данных по USART.
%		\end{enumerate}
%		\item Инициализация дисплея по интерфейсу SPI1:
%		\begin{enumerate}
%			\item Настройка интерфейса GPIO.
%			\item Настройка SPI на запись.
%			\item Настройка DMA канала.
%			\item Инициализация дисплея.
%		\end{enumerate}
%		\item Вход в бесконечный цикл, ожидания команды по USART.
%		\item При получении команды выполняется соответствующая команде действие.
%	\end{enumerate}
	
%	\textbf{Инициализация модуля RS232.}
	
%	\begin{enumerate}
%		\item Проверка подколюченных модулей, выбор соответвствующего модуля.
%		\item Настройка работы USART.
%		\item Считывание характеристик дисплея: разрешения, модель.
%	\end{enumerate}
	
%	\textbf{Алгоритм обновления кадра.}
	
%	Компьютер отправляет на микроконтроллер команду для обновления кадра на дисплеи. После получения от микроконтроллера подтверждения готовности работы происходит считывание снимка, который обновить кадр на дисплеи. Происходит отправка первой строчки, ожидание получения строки. Отправка второй строки, третей и т. д.
	
%	После завершения передачи происходит очистка памяти от снимка.
	
%	\includeimage{06-block_5}{f}{H}{0.55\textwidth}
%	{Алгоритм работы микроконтроллера, когда на него поступает команда обновления кадра.} % Подпись рисунка
	
%	Алогоритм работы микроконтроллера в момент выполнения операции по обновлению кадра описан на рисунке \ref{img:06-block_5}.
	
%	Важно добавить, что при больших скоростях возникает искажение данных из за дребезга контактов. Чтобы этого избежать все соединения интерфейсов должны быть спаяны.
	
	