\chapter{Алгоритм работы микроконтроллера}

	В предыдудущей главе была описана обзая схема работы монитора. Но в начале разберем общие настройки и инициалиазию микроконтроллера и модуля RS232.
	
	\textbf{Инициализация микроконтроллера.}
	
	\begin{enumerate}
		\item Настройка частоты работы. Ядро настраивается на частоту 168 МГц, шина AHB/APB1 на частоту 82 МГц, AHB/APB2 42 МГц.
		\item Инициализация USART2:
		\begin{enumerate}
			\item Получение системной частоты.
			\item Настройка интерфейса GPIO.
			\item Настройка USART на чтение и запись.
			\item Настройка DMA канала.
			\item Включения прерываний на чтение данных по USART.
		\end{enumerate}
		\item Инициализация дисплея по интерфейсу SPI1:
		\begin{enumerate}
			\item Настройка интерфейса GPIO.
			\item Настройка SPI на запись.
			\item Настройка DMA канала.
			\item Инициализация дисплея.
		\end{enumerate}
		\item Вход в бесконечный цикл, ожидания команды по USART.
		\item При получении команды выполняется соответствующая команде действие.
	\end{enumerate}
	
	\textbf{Инициализация модуля RS232.}
	
	\begin{enumerate}
		\item Проверка подколюченных модулей, выбор соответвствующего модуля.
		\item Настройка работы USART.
		\item Считывание характеристик дисплея: разрешения, модель.
	\end{enumerate}
	
	\textbf{Алгоритм обновления кадра.}
	
	Компьютер отправляет на микроконтроллер команду для обновления кадра на дисплеи. После получения от микроконтроллера подтверждения готовности работы происходит считывание снимка, который обновить кадр на дисплеи. Происходит отправка первой строчки, ожидание получения строки. Отправка второй строки, третей и т. д.
	
	После завершения передачи происходит очистка памяти от снимка.
	
	\includeimage
	{06-block_5}{f}{H}{0.55\textwidth}
	{Алгоритм работы микроконтроллера, когда на него поступает команда обновления кадра.} % Подпись рисунка
	
	Алогоритм работы микроконтроллера в момент выполнения операции по обновлению кадра описан на рисунке \ref{img:06-block_5}.
	
	Важно добавить, что при больших скоростях возникает искажение данных из за дребезга контактов. Чтобы этого избежать все соединения интерфейсов должны быть спаяны.
	
	