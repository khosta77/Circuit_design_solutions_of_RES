\chapter*{ЗАКЛЮЧЕНИЕ}
\addcontentsline{toc}{chapter}{ЗАКЛЮЧЕНИЕ}

	В данной работе я привел основной способ вывода изображений на графический дисплей. Полученное устройство позволяет тестировать различные дисплеи на качество выводимого ими изображения.
	
	В ходе работы трудность, с которой я предполагал столкнутся оказалась не значительной. Главной проблемой оказались разные цветовые форматы различных дисплеев, из-за этого оказалось не возможным разработать универсальное устройство способное работать с любыми дисплеями, без предварительной настройки. 
	
	В будущей версии устройства я планирую решить эту проблему, чтобы можно было подключать любой дисплей и он начинал работать без предварительной программной настройки. Так же изготовление печатной платы для дисплея будет способствовать исклюению дребезга контактов. В заказанной печатной плате будет расположен модуль для внешней памяти, который позволит расширить поддерживаемое разрешение.

%	Результатом работы стал компактный монитора, который подключается к любому компьютеру. 
	
%	Из недостатков устройства выделяется:
	
%	\begin{enumerate}
%		\item Медленную скорость работы декодера, интерфейс передачи данных имеет ограничения по скорости работы. А при больших скоростях появляются потери данных.
%		\item Для компьютера трубуется драйвер, чтобы работать с дисплеем.
%		\item При малом дисплеи не большая скорость работы интерфейса SPI ещё не является узким место в работе, но при большом разрешение дисплея нужно будет использовать другой стандарт для работы с ним, для более высокой частоты кадров.
%	\end{enumerate}

%	В будущих версиях устройства данные недостатки можно устранить. Так же можно добавить новые возможности.
	
%	\begin{enumerate}
%		\item Добавить более быстрый декодер, либо использовать интерфейс подключения USB на прямую с микроконтроллером.
%		\item Возможность дополнительного подключения по интерфейсу SPI, LVDS для работы с микрокомпьютерами в более удобном виде.
%		\item Изготовление печатной платы, чтобы исключить дребезг контактов.
%		\item Повышение скорости работы.
%		\item Реализовать Touch Display.
%	\end{enumerate}
