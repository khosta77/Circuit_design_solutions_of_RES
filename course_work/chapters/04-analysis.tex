\chapter{Обзор существующих решений}

	Компактные мониторы делятся на группы:
	
	\begin{enumerate}
		\item Для персональных компьютеров. Используют для подключения интерфейсы HDMI, DP и т. д. Представляют из себя обычные мониторы, с меньшей диагональю.
		\item Для микрокомпьютеров на подобии Rasberry Pi. Подключаются по выведенному на плате интерфейсу ввода/вывода. Диагональ не больше четырех дюймов.
	\end{enumerate}
	
	 Моя реализация компактного монитора будет схожа со второй группой. Интерфейс HDMI является закрытым и использовать его возможно только по лицензии, реализация монитора соответствующей первой группе будет затруднительной.
	 
	 Мониторы для микрокомпьютеров используют интерфейсы для передачи данных на дисплей: SPI, LVDS, LTDS и т. д.
	 
	 \includeimage
	 {04-General_principle_of_operation_of_a_compact_display}{f}{H}{0.6\textwidth}
	 {Алгоритм работы компактного дисплея} % Подпись рисунка
	 
	 В приближенном виде с компьютера по одному из возможных интерфейсов поступает сигнал о начале передаче данных. В ходе передачи происходит декодирование данных и последующая передача их в микроконтроллер. Который обновляет кадр на дисплее.
	 
	 В некоторых случаях декодер может быть встроен в микроконтроллер дисплея.