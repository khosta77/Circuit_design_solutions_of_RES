\chapter{Исследовательский раздел}

	Есть группа дисплеев: 16-битные ST7735 и ILI9348, 18-битный ILI9488. Предлагается проверить качество вывода цвета у данных дисплеев.%Возникает вопрос: какое качество вывода цвета у данных дисплеев? %Вопрос: А какое качество вывода цвета у данных дислпеев?
	
	Так как данная проверка и оценка не явлется стандартизированной. То методикой проверки качества дисплея будет фотографирование дисплеев по очередно в объектив камеры "<Apple iphone 12"> так, чтобы в центре снимка был дисплей. Камера была по центру дисплея, на расстоянии примерно 10 сантиметров. Уровень освещенности исходил только от дисплеев, эксперемент проходил без внешних источников света. Снимки будут перемещены на компьютер и по пиксельно проверены в программе "<Paint"> с помощью инструмента "<Пипетка">.
	
	Эталоными значениями для камеры будет снимок экрана компьютера macbook. Тестовыми данными будет снимок с 6-ю цветами: красным(0xFF0000), зеленым(0x00FF00), синим(0x0000FF), желтым(0xFFFF00), фиолетовым(0xFF00FF), бирюзовым(0x00FFFF).
	
	Важным будет уточнить, что существуют разные способы проверки качества дисплеев. Использование специальной аппаратуры, устройство, которое прилоняется к дисплею и измеряет цветовой спектр. Специальные от калиброванные камеры, для 100\% передачи цвета. Ввиду отсутсвия данной аппаратуры был предолжен способ описанный выше. Не достатоком его может служить камера "<Apple iphone 12"> так как с завода она может иметь откланения, которые добавляют насыщенности снимку, что вносит искажение в полученный результат. 
	
	\includeimage{07-ILI9341_ST7789_test_240x320}{f}{H}{0.3\textwidth}
	{Пример тестового изображениея.}
	
	Проведем измерения, результат занесем в таблицу:
	
	\includeimage{07-all-photo}{f}{H}{1.0\textwidth}
	{Все измерения цвета. Слева на право: Экран Macbook, ILI9488, ILI9341, ST7735.}
	
	\begin{tabular}{ |p{6cm}||p{2cm}|p{2cm}|p{2cm}|  }
		\hline
							   & Red & Green & Blue \\ \hline
		\multicolumn{4}{|c|}{Эталон, экран macbook} \\ \hline
		Зона красного цвета    & 255 & 0     & 0    \\
		Зона зеленого цвета    & 63  & 243   & 0    \\
		Зона синего цвета      & 0   & 0     & 251  \\
		Зона желтого цвета     & 245 & 250   & 0    \\
		Зона фиолетового цвета & 250 & 0     & 247  \\
		Зона бирюзового цвета  & 48  & 218   & 217  \\ \hline
		\multicolumn{4}{|c|}{ST7735} \\ \hline
		Зона красного цвета    & 245 & 141   & 168  \\  % Какой-то розовый выходит
		Зона зеленого цвета    & 156 & 255   & 211  \\  % Из далеко еще похож на зеленый
		Зона синего цвета      & 0   & 42    & 255  \\  % Корректный цвет
		Зона желтого цвета     & 223 & 233   & 194  \\  % Бледно-желтый
		Зона фиолетового цвета & 180 & 131   & 255  \\
		Зона бирюзового цвета  & 157 & 255   & 255  \\ \hline  % Из далеко белый
		\multicolumn{4}{|c|}{ILI9341} \\ \hline
		Зона красного цвета    & 158 & 55    & 63   \\  % Какой-то бордовый
		Зона зеленого цвета    & 0   & 156   & 90   \\  % К желтому близок
		Зона синего цвета      & 0   & 28    & 255  \\  % Корреткный
		Зона желтого цвета     & 134 & 201   & 142  \\  % По цвету близок зеленому
		Зона фиолетового цвета & 104 & 53    & 255  \\  % Корректный более-менее
		Зона бирюзового цвета  & 0   & 161   & 255  \\ \hline
		\multicolumn{4}{|c|}{ILI9488} \\ \hline
		Зона красного цвета    & 139 & 67    & 90   \\  % Какой-то темнокрасный
		Зона зеленого цвета    & 10  & 229   & 87   \\  % В близи не корректный, из далека зеленый
		Зона синего цвета      & 15  & 18    & 255  \\  % Корректный
		Зона желтого цвета     & 178 & 233   & 176  \\  % белый какойто на выходе
		Зона фиолетового цвета & 160 & 117   & 255  \\
		Зона бирюзового цвета  & 0   & 186   & 250  \\ \hline % Корректный
	\end{tabular}\\

	По результатам измерений можем сделать промежуточные вывод.
	
	Эталонные значения близки к заданным. Отклонения могут быть связаны с неидеальными условиями эксперимента: Внешние источники света, не идеально чисты экран, колебания расстояний. Из этого следует, что камера давала в дальнейших тестах не знанчительные отклоненения.
	
	ST7735: зеленый и красный были переданы дисплеем слишком яркими, с наполненностью всех цветов, в отличии от синего, который едниственный близок к эталонном значениям. Желтый цвет близок к белому цвету, не корректная передача. Фиолетовый с более глубоким синим цветом, другие компоненты не полные из за этого фиолетовый выходит затемненным. Бирюзовый цветы слишком близок белому цвету.

	ILI9341 и ILI9488: Результаты дисплеев схожи между собой. Корректно передается синий цвет. На ILI9488 красный выходит чуть тусклее, чем на ILI9341. Остальные цвета схожи между собой, разница возникает из погрешности измерений.
	
	Для глаза 3 протестированных дисплея выходят по качеству одинаковыми.

%	При запуске программы с дисплеем ST7735 настроенным на 16-битный режим работы на монитор была выведена картинка с потерянными цветами.
	
%	\includeimage{07-no-cool_16_bit}{f}{H}{0.4\textwidth}
%	{Не корректное тестовое изображение в 16-битном формате.} % Подпись рисунка
	
%	Была выдвинута гипотезака, что проблемой некорректной цвето передачи была не правильная формула конвертации увета из RGB888 в RGB565.
	
%	\[\text{RGB565} = ((\text{R } \& \text{ 0b}11111000) << 8) | ((\text{G } \& \text{ 0b}11111100) << 3) | (\text{B} >> 3)\]
	
%	Но проверив отрисовку цвета по этой формуле без привязки к работе с компьютером было подтверждено, что формула корректная. В качестве проверки себя, были перепроверены результаты на выходе при входе разных цветов. То есть на вход G был подан R или B и аналогично для других. Отдельно на компьютере был проверен алгоритм чтения и записи снимков, который работал корректно.
	
%	Для проверки перешел с 16-битного дисплея на 18-битный дисплей ILI9488.
	
%	\includeimage{07-no-cool_18_bit}{f}{H}{0.33\textwidth}
%	{Не корректное тестовое изображение в 18-битном формате.} % Подпись рисунка
	
%	Результатом на выходе работы устройства была от зеркальная фотография, в которой были не корректные цвета. После того как в программу были добавлены строчки кода, которые при считывании снимка производили его о тзеркаливание и разворот цвета, то картинка вывелась успешно.
	
%	\includeimage{07-cool_18_bit}{f}{H}{0.33\textwidth}
%	{Исправленное тестовое изображение в 18-битном формате.} % Подпись рисунка
	
%	И соответвственно в 16-битном формате тоже был корректный вывод.
	
%	\includeimage{07-cool_16_bit}{f}{H}{0.4\textwidth}
%	{Исправленное тестовое изображение в 16-битном формате.} % Подпись рисунка
	
%	В результате исследования является уточнение особенности работы пары микроконтроллер-декодер где на вход модуля должны приходить данные в от зеркальном виде.
	
%	Готовый код проекта монитора выложен в открытом доступе на сайте github в отдельном \href{https://github.com/khosta77/stm32f4_ILI9488}{репозитории для физического устройства}. Для \href{https://github.com/khosta77/ft2xx_fast_start}{драйвера} на компьютер.
	
	